\documentclass[a4paper,12pt]{article}
\usepackage[colorlinks=true,
	    linkcolor=blue,
    	    urlcolor=blue]{hyperref}
\usepackage{listings}
\usepackage{color}


\definecolor{codegreen}{rgb}{0,0.6,0}
\definecolor{codegray}{rgb}{0.5,0.5,0.5}
\definecolor{codepurple}{rgb}{0.58,0,0.82}
\definecolor{backcolour}{rgb}{0.95,0.95,0.92}

\lstdefinestyle{mystyle}{
	backgroundcolor=\color{backcolour},   
	commentstyle=\color{codegreen},
	keywordstyle=\color{magenta},
	numberstyle=\tiny\color{codegray},
	stringstyle=\color{codepurple},
	basicstyle=\footnotesize,
	breakatwhitespace=false,         
	breaklines=true,                 
	captionpos=b,                    
	keepspaces=true,                 
	numbers=left,                    
	numbersep=5pt,                  
	showspaces=false,                
	showstringspaces=false,
	showtabs=false,                  
	tabsize=2
}
\lstset{style=mystyle}






\title{QDG Raspberry Pi Ambient Sensing Project}
\date{\today}
\author{Tristan Calderbank}

\begin{document}
\maketitle
\tableofcontents
\newpage

\section{Raspberry Pi}

\subsection{Setup}

\sloppy
\begin{enumerate}

\item Format the Pi's SD card and install Raspian (a linux distro) using instructions from the Raspberry Pi website: \href{https://www.raspberrypi.org/help/noobs-setup/}{https://www.raspberrypi.org/help/noobs-setup/}
\item Start up the Raspberry Pi and login with the default username: \textbf{pi} and password: \textbf{raspberry}

\item Connect to the internet and run system updates with the following terminal commands:

\begin{lstlisting}[language=bash]
sudo apt-get update
sudo apt-get upgrade
\end{lstlisting}

\item Set the system time using this command:

\item Navigate to the user home directory and clone the source code repo from github:
\begin{lstlisting}[language=bash, mathescape]
cd $\sim$
git clone https://github.com/tristancalderbank/qdg-pi
\end{lstlisting}


\end{enumerate}

\subsection{Redis Server}

In order for the Raspberry Pis to send data back to the lab computer we use a server called Redis. The developers recommend compiling it manually rather than using apt-get.

\begin{enumerate}

\item Run the following commands to install Redis:
\begin{lstlisting}[language=bash, mathescape]
cd /home/pi/Downloads
wget http://download.redis.io/redis-stable.tar.gz
tar xvzf redis-stable.tar.gz
cd redis-stable
make
sudo cp src/redis-server /usr/local/bin/
sudo cp src/redis-cli /usr/local/bin/
sudo mkdir /etc/redis
sudo mkdir /var/redis
sudo cp utils/redis_init_script /etc/init.d/redis_6379
sudo cp /home/pi/qdg-pi/6379.conf /etc/redis/6379.conf
sudo update-rc.d redis_6379 defaults
\end{lstlisting}

\end{enumerate}

\subsection{Temperature, Pressure, and Humidity Sensors}

The Pi uses a 3-in-one sensor made by Adafruit called the BME280. The datasheet can be found here: \href{https://www.adafruit.com/datasheets/BST-BME280\_DS001-10.pdf}{https://www.adafruit.com/datasheets/BST-BME280\_DS001-10.pdf}

\begin{enumerate}

\item The sensor uses this driver: \href{https://github.com/adafruit/Adafruit\_Python\_BME280}{https://github.com/adafruit/Adafruit\_Python\_BME280} which is already included with the project source code for this project
\item Before using the driver you must first install the Adafruit Python GPIO package by running the following commands:

\begin{lstlisting}[language=bash]
sudo apt-get install build-essential python-pip python-dev python-smbus git
cd /home/pi/Downloads
git clone https://github.com/adafruit/Adafruit_Python_GPIO.git
cd Adafruit_Python_GPIO
sudo python setup.py install
\end{lstlisting}

\item The sensor communicates using the I2C protocol which we must first enable on the Raspberry Pi. Use this guide to enable I2C: \href{https://learn.adafruit.com/adafruits-raspberry-pi-lesson-4-gpio-setup/configuring-i2c}{https://learn.adafruit.com/adafruits-raspberry-pi-lesson-4-gpio-setup/configuring-i2c}

\end{enumerate}

\subsection{Python Scripts}

\begin{enumerate}
\item First install the python Redis module as well as the timezone module:
\begin{lstlisting}[language=bash]
sudo pip install redis
sudo pip install pytz
\end{lstlisting}
\item To make the data publishing script run on startup we add it to Crontab:
\begin{lstlisting}[language=bash]
crontab -e
\end{lstlisting}
\item Add the following line to the end of the crontab file:
\begin{lstlisting}[language=bash]
@reboot redis-server &
@reboot sudo python /home/pi/qdg-pi/publish-data.py
\end{lstlisting}

\item Go into the qdg-pi folder and edit the script called "publish\_data.py" to have the correct IP addresses for the Pi's


\end{enumerate}

\section{Host Computer}
\subsection{Recieving Data from the Pi's}

\subsection{Hosting the Web App}

\begin{enumerate}

\item First install apache and php to host the page:

\begin{lstlisting}[language=bash]
sudo apt-get install apache2
sudo apt-get install php5
\end{lstlisting}

You can make sure it worked by typing your local ip address into a browser. You should get the default apache2 webpage.

\item Navigate to the default apache server directory and delete the folder called "html". Clone the web app repo and rename the folder "html":

\begin{lstlisting}[language=bash]
cd /var/www/
sudo rm -rf html/
sudo git clone https://github.com/tristancalderbank/ambient-data
sudo mv ambient-data html
\end{lstlisting}

\item Configure permissions for apache to read/write files on the server:
\begin{lstlisting}[language=bash]
sudo chgrp -R www-data html
sudo chmod -R 755 html
\end{lstlisting}

\end{enumerate}

\end{document}






















