\documentclass[a4paper,12pt]{article}
\usepackage{hyperref}
\usepackage{breakurl}
\usepackage{listings}
\usepackage{color}


\definecolor{codegreen}{rgb}{0,0.6,0}
\definecolor{codegray}{rgb}{0.5,0.5,0.5}
\definecolor{codepurple}{rgb}{0.58,0,0.82}
\definecolor{backcolour}{rgb}{0.95,0.95,0.92}

\lstdefinestyle{mystyle}{
	backgroundcolor=\color{backcolour},   
	commentstyle=\color{codegreen},
	keywordstyle=\color{magenta},
	numberstyle=\tiny\color{codegray},
	stringstyle=\color{codepurple},
	basicstyle=\footnotesize,
	breakatwhitespace=false,         
	breaklines=true,                 
	captionpos=b,                    
	keepspaces=true,                 
	numbers=left,                    
	numbersep=5pt,                  
	showspaces=false,                
	showstringspaces=false,
	showtabs=false,                  
	tabsize=2
}
\lstset{style=mystyle}






\title{QDG Raspberry Pi Ambient Sensing Project}
\date{\today}
\author{Tristan Calderbank}

\begin{document}
\maketitle
\tableofcontents

\section{Setting up the Raspberry Pi OS}

\sloppy
\begin{enumerate}

\item Format the Pi's SD card and install Raspian (a linux distro) using instructions from the Raspberry Pi website: \href{https://www.raspberrypi.org/help/noobs-setup/}{https://www.raspberrypi.org/help/noobs-setup/}
\item Start up the Raspberry Pi and login with the default username: \textbf{pi} and password: \textbf{raspberry}

\item Connect to the internet and run system updates with the following terminal commands:

\begin{lstlisting}[language=bash]
sudo apt-get update
sudo apt-get upgrade
\end{lstlisting}

\end{enumerate}

\section{Redis Server}

In order for the Raspberry Pi's to send data back to the lab computer we use a server called Redis.

\begin{enumerate}

\item Run the following command to install Redis:
\begin{lstlisting}[language=bash]
sudo apt-get install redis-server
\end{lstlisting}



\end{enumerate}

\section{Temperature, Pressure, and Humidity sensors}

The Pi uses a 3-in-one sensor made by Adafruit called the BME280. The datasheet can be found here: \href{https://www.adafruit.com/datasheets/BST-BME280\_DS001-10.pdf}{https://www.adafruit.com/datasheets/BST-BME280\_DS001-10.pdf}

\begin{enumerate}

\item The sensor uses this driver: \href{https://github.com/adafruit/Adafruit\_Python\_BME280}{https://github.com/adafruit/Adafruit\_Python\_BME280} which is already included with the project source code for this project
\item Before using the driver you must first install the Adafruit Python GPIO package by running the following commands:

\begin{lstlisting}[language=bash]
sudo apt-get update
sudo apt-get install build-essential python-pip python-dev python-smbus git
git clone https://github.com/adafruit/Adafruit_Python_GPIO.git
cd Adafruit_Python_GPIO
sudo python setup.py install
\end{lstlisting}

\item The sensor communicates using the I2C protocol which we must first enable on the Raspberry Pi. Use this guide to enable I2C: \href{https://learn.adafruit.com/adafruits-raspberry-pi-lesson-4-gpio-setup/configuring-i2c}{https://learn.adafruit.com/adafruits-raspberry-pi-lesson-4-gpio-setup/configuring-i2c}

\end{enumerate}

\end{document}






















